\documentclass[10pt,a4paper,landscape,twocolumn]{scrartcl}
\usepackage[utf8]{inputenc}
\usepackage{amsmath}
\usepackage{amsfonts}
\usepackage{amssymb}
\usepackage{graphicx}

%Additional commands
\newcommand{\desc}[2]{\noindent #1 - \textit{#2} \\}
\newcommand{\usage}[1]{ Usage: \\ \texttt{#1} \\}
\newenvironment{examplebox}{e. g.: \\}{}
\newcommand{\example}[2]{\texttt{#1} \\ \textit{#2} \\}
\newcommand{\vm}[1]{\textbf{#1} vm}
\newenvironment{cmdblock}{}{\\}
\newcommand{\n}{$\hookleftarrow$ \\ }

%
\title{Qubes OS Cheatsheet}
\begin{document}
\maketitle
\subsection*{VM Management}
\begin{cmdblock}
	\desc{qvm-start}{starts a vm}
	\usage{qvm-start [options] $<$vm-name$>$}
	\begin{examplebox}
		\example{qvm-start personal}{starts the \vm{personal}}
		\example{qvm-start ubuntu \n --cdrom personal:/home/user/ \n Downloads/ubuntu-14.04.iso}{starts the \vm{ubuntu} with cdrom}
	\end{examplebox}
\end{cmdblock}

\begin{cmdblock}
	\desc{qvm-run}{runs a specific command on a vm}
	\usage{qvm-run [options] [$<$vm-name$>$] [$<$cmd$>$]}
	\begin{examplebox}
		\example{qvm-run personal xterm}{runs xterm on \vm{personal}}
		\example{qvm-run personal xterm --pass-io}{runs xterm and passes all sdtin/stdout/stderr to the terminal}
		\example{qvm-run personal "sudo yum update" --pass-io --nogui}{pass a specific command directly to the VM}
	\end{examplebox}
\end{cmdblock}

\begin{cmdblock}
	\desc{qvm-block}{list/set VM PCI devices}
	\usage{
		qvm-block -l [options]\\
		qvm-block -a [options] $<$device$>$ $<$vm-name$>$\\
		qvm-block -d [options] $<$device$>$\\
		qvm-block -d [options] $<$vm-name$>$
	}
	\begin{examplebox}
		\example{qvm-block -A personal dom0:/home/user/extradisks/data.img}{attaches an additional storage for the \vm{personal} }
	\end{examplebox}
\end{cmdblock}

\begin{cmdblock}
	\desc{qvm-prefs}{list/set various per-VM properties}
	\usage{
		qvm-prefs -l [options] $<$vm-name$>$\\
		qvm-prefs -s [options] $<$vm-name> $<$property$>$ [...]
	}
	\begin{examplebox}
		\example{qvm-prefs win7-copy}{lists the preferences of the \vm{win7-copy} }
		\example{qvm-prefs win7-copy -s mac 00:16:3E:5E:6C:05}{set a new network cards mac}
		\example{qvm-prefs lab-win7 -s qrexec\textunderscore installed true}{sets the qrexec installed preference}
		\example{qvm-prefs lab-win7 -s qrexec\textunderscore timeout 120}{usefull for windows hvm based vms}
		\example{qvm-prefs lab-win7 -s default\textunderscore user joanna}{sets the login user}
	\end{examplebox}
\end{cmdblock}

\begin{cmdblock}
	\desc{qvm-ls}{list VMs and various information about their state}
	\usage{qvm-ls [options] $<$vm-name$>$}
	\begin{examplebox}
		\example{qvm-ls}{lists all vms}
	\end{examplebox}
\end{cmdblock}

\begin{cmdblock}
	\desc{qvm-sync-appmenus}{updates desktop file templates for given StandaloneVM or TemplateVM}
	\usage{qvm-sync-appmenus [options] $<$vm-name$>$}
	\begin{examplebox}
		\example{qvm-sync-appmenus archlinux-template}{useful for custom .desktop files or distributions not using yum}
	\end{examplebox}
\end{cmdblock}

\begin{cmdblock}
	\desc{qvm-ls}{list VMs and various information about their state}
	\usage{qvm-ls [options] $<$vm-name$>$}
	\begin{examplebox}
		\example{qvm-ls -n}{Show network addresses assigned to VMs}
		\example{qvm-ls -d}{Show VM disk utilization statistics}
	\end{examplebox}
\end{cmdblock}

\subsection*{Dom0}
\begin{cmdblock}
	\desc{qubes-dom0-update}{updates software in dom0}
	\usage{ qubes-dom0-update [--clean][--check-only][--gui] [$<$yum opts$>$][$<$pkg list$>$]}
	\begin{examplebox}
		\example{sudo qubes-dom0-update}{updates dom0}
		\example{sudo qubes-dom0-update qubes-windows-tools}{install the windows tools}
	\end{examplebox}
\end{cmdblock}

\subsection*{Copy from \& to Dom0}
Copy from Dom0 to VM:\\
\begin{verbatim}
cat /path/to/file_in_dom0 | 
 qvm-run --pass-io <dst_domain> 
  'cat > /path/to/file_name_in_appvm'
\end{verbatim}
Copy from VM to Dom0:\\
\begin{verbatim}
qvm-run --pass-io <src_domain> 
 'cat /path/to/file_in_src_domain' >
  /path/to/file_name_in_dom0
\end{verbatim}

\subsection*{DomU}

\subsection*{Shortcuts}
Copy text between VM A and B.\\
\textit{On VM A (source):}
\begin{enumerate}
	\item CTRL+C
	\item CTRL+SHIFT+C
\end{enumerate}
\textit{On VM B (destination):}
\begin{enumerate}
	\item CTRL+SHIFT+V
	\item CTRL+V
\end{enumerate}

\subsection*{Grow disk}
sudo resize2fs /dev/xvdb

\subsection*{VM $\leftrightarrow$ VM Networking}
Make sure:
\begin{itemize}
	\item Both VMs are connected to the same firewall VM
	\item Qubes IP addresses are assigned to both VMs
	\item Both VMs are started
\end{itemize}
\subsubsection*{For the current session}
In the firewall VM's terminal:
\begin{verbatim}
sudo iptables -I FORWARD 2 -s <IP address of A> 
  -d <IP address of B> -j ACCEPT
\end{verbatim}

\end{document}

